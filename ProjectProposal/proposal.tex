% this is a comment in latex
% substitute this documentclass definition for uncommented one
% to switch between single and double column mode
%\documentclass[11pt,twocolumn]{article}
\documentclass[11pt]{article}

% use some other pre-defined class definitions for the style of this
% document.   
% The .cls and .sty files typically contain comments on how to use them 
% in your latex document.  For example, if you look at psfig.sty, the 
% file contains comments that summarize commands implemented by this style 
% file and how to use them.
% files are in: /usr/share/texlive/texmf-dist/tex/latex/preprint/
\usepackage{fullpage}
\usepackage{subfigure,indentfirst}
% for url
\usepackage{hyperref}
% for underlined text
\usepackage[normalem]{ulem}

% use some packages for importing figures of different types
% pdfig is one for importing .pdf files.  sadly, they are not all
% compatible, so you often have to convert figures to the same type.
\usepackage{epsfig,graphicx}


% you can also define your own formatting directives.  I don't like
% all the space around the itemize and enumerate directives, so
% I define my own versions: my_enumerate and my_itemize
\newenvironment{my_enumerate}{
  \begin{enumerate}
    \setlength{\itemsep}{1pt}
      \setlength{\parskip}{0pt}
\setlength{\parsep}{0pt}}{\end{enumerate}
}

\newenvironment{my_itemize}{
  \begin{itemize}
    \setlength{\itemsep}{1pt}
      \setlength{\parskip}{0pt}
\setlength{\parsep}{0pt}}{\end{itemize}
}

% this starts the document
\begin{document}

% for an article class document, there are some pre-defined types
% for formatting certain content: title, author, abstract, section

\title{CS87 Project Proposal: A Concise and Meaningful Title}

\author{Author1, Author2, Author3 \\ 
Computer Science Department, Swarthmore College, Swarthmore, PA  19081}

\maketitle

\section {Introduction}\label{intro} 
A 1-2 paragraph summary of the problem you are solving, why it is interesting,
how you are solving it, and what conclusions 
you expect to draw from your work.

\section {Related Work}\label{rel}
1-2 paragraphs describing similar approaches to the one you propose. This need
not be an exhaustive summary of related literature, but should be used to put
your solution in context and/or to support your solution. This is also a good
way to motivate your work. This can be a summary taken from your longer
annotated bibliography.  

Here is an example of how to cite someting from the bib 
file~\cite{newhall:nswap2L}.  Here is another~\cite{unixV}.  
The proposal.bib file has some example 
bibtex entries you can use as a guide for entering your own.

In the Annotated Bibliography~\ref{annon} you will included an expanded description of your related work (and you should cite there as well.

\section {Your Solution}\label{soln}
3-4 paragraphs describing what you plan to do, how you plan to do it, how it
solves the problem, and what types of conclusions you expect to draw from your
work.

\section {Experiments}\label{exper}
1-3 paragraphs describing how you plan to evaluate your work. List the
experiments you will perform. For each experiment, explain how you will perform
it and what the results will show (explain why you are performing a particular
test).

\section {Equipment Needed}\label{equip}
1 paragraph listing any software tools that you will need to implement and/or
test your work. If you need to have software installed to implement your
project, you should check with the systems lab to see if it is something that
can be installed on the CS lab machines.

\section {Schedule}\label{sched}
list the specific steps that you will take to complete your project, include
dates and milestones. This is particularly important to help keep you on track,
and to ensure that if you run into difficulties completing your entire project,
you have at least implemented steps along the way. Also, this is a great way to
get specific feedback from me about what you plan to do and how you plan to do
it.  Conclusions: 1 paragraph summary of what you are doing, why, how, and what
you hope to demonstrate through your work.

% here is an example of a numbered list 
\begin{my_enumerate}
  \item Week 1: 
  \item Week 2: 
\end{my_enumerate} 



% The References section is auto generated by specifying the .bib file
% containing bibtex entries, and the style I want to use (plain)
% compiling with latex, bibtex, latex, latex, will populate this
% section with all references from the .bib file that I cite in this paper
% and will set the citations in the prose to the numbered entry here
\bibliography{proposal}
\bibliographystyle{plain}

% force a page break
\newpage 

% I want the Annotated Bib to be single column pages
\onecolumn
\section*{Annotated Bibliography}\label{annon} 

This section does not count towards the page total for your proposal.


\end{document}

